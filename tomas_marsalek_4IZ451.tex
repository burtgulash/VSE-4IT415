\documentclass[11pt]{article}
\usepackage[utf8]{inputenc}
\usepackage[english]{babel}
\usepackage{graphicx}
\usepackage[numbers]{natbib}

\graphicspath{{pics/}}

\DeclareUnicodeCharacter{00A0}{ }


\title{Funkce repository návrhu systému a důvody pro její existenci. \\ 4IT415 Informační modelování organizací (ZS 2014-2015)}
\author{Tomáš Maršálek}
\date{\today}

\begin{document}
\maketitle
\thispagestyle{empty}
\clearpage

\section{Úvod}
Repository pattern je návrhový vzor s účelem logického oddělení perzistenční
vrstvy systému od aplikační. Do systému přináší nízkou provázanost (loose
coupling), což je obecně považováno jako dobrá vlastnost při návrhu. Nízká
provázanost umožňuje nahrazení části s minimálním zásahem do jiných částí
systému. Proč bychom ale v první řadě uvažovali o nahrazování perzistenční
vrstvy?

Prvním důvodem může být nahrazení implementace úložiště. Repository nám dává
možnost snadné migrace mezi různými databázemi. Nebo pokud chceme nahradit
pouze část úložiště, např. když chceme ukládat obrázky do souborů a ne přímo do
databáze, aplikační kód se o téhle změně vůbec nemusí dozvědět, protože k
obrázkům přistupuje pouze definovaných metod rozhraní repozitáře.

Změna implementace úložiště je ale většinou uváděna pouze jako ilustrativní
využití repository, protože ke změnám implementace v produkčním systému dochází
zřídkakdy. Většinou existuje datová základna a aplikace se podle ní
přizpůsobují než naopak.

Dalším, a to podstatnějším, důvodem je možnost používat unit testy pro
aplikační kód, který je provázaný perzistenčním kódem, a ne jen integrační
testy, které testují aplikaci jako celek. Repository nám totiž umožňuje
vytvořit takzvanou Mock implementaci perzistenčního rozhraní, díky které je
možné jednoduše vytvořit sadu deterministických unit testů. Unit testy se
skutečnou databází jsou velmi krkolomné, protože před každým testem je třeba
testovací databázi vyprázdnit a deterministicky naplnit daty potřebnými pro
konkrétní unit test.

Repository pattern není často oblíbený, protože není flexibilní. Rozhraní
repository pak může obsahovat metodu pro každou permutaci získání objektu
(např. getUserById, getUserByName, getUsersWithDetails,
getUsersWithDetailsAndFavouriteColorByAddress). Tento problém spadá do většího
problému zvaného object-relational impedance mismatch, který ve zkratce
znamená, že je velmi obtížné přesně namapovat objekty z OO světa do relačního
modelu. Doposud nikdo nepřišel s řešením, které by bylo vhodné pro všechny
případy a stále by mělo vlastnosti nízkého provázání perzistenční a aplikační
vrstvy. Problém dostal přezdívku \uv{Vietnam of Computer Science} v analogii s
Vietnamskou válkou, do které se dlouhé roky investovaly neuvěřitelné peníze a
výsledek nikde. 

Repository je jedním z řešení tohoto problému s vlastností nizkého provázání,
avšak za cenu velmi složitého rozhraní, jak bylo zmíněno výše. Jediným uváděným
řešením jak se zbavit O/R mapování je odstranit z názvu \uv{O} nebo \uv{R},
tedy nepoužívat objekty v aplikaci, anebo nepoužívat relační databazi, ale
například objektovou nebo grafovou NoSQL. Dalšími navrhovanými řešeními jsou
experimentální Event Sourcing (Akka persistence ve Scale, acid-state v
Haskellu),

Řešení, které se v poslední době s rozšířením architektury MVC pro webové
aplikace stalo populární, je takzvané ORM - Object Relational Mapper, neboli
nástroj, který provede konverzi dat z tabulek relační databáze do grafové
reprezentace objektů v objektově orientovaném programování. Vývojář se nesetká
s neduhy jako u repository, kde musel vytvořit každou permutaci databázového
dotazu, ale tvoří databázové dotazy přímo v kódu podle potřeby. Nevytváří je
ale v jazyku závislém na implementaci databáze, ale v konstrukcích nezávislých
na implementaci. Alespoň tohle byl původní záměr a bohužel skutečnost je jiná.
ORM přináší pouze výhodu v tom, že se v aplikační vrstvě nemusí používat přímo
databázový dotazovací jazyk, ale jeho varianta, která odpovídá konstrukcím
programovacího jazyka aplikační vrstvy. Nedocílíme nízkého provázání, nelze
vytvářet Mocky (implementace ORM nástrojů je hutná a vytvořit mock, který by
splňoval jejich rozhraní je velmi obtížné)

\section{Návrh API}
\section{Caching}
\section{CQRS}


\bibliographystyle{csplainnat}
\bibliography{ref}

\end{document}
