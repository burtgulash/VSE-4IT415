\documentclass[10pt]{article}
\usepackage[utf8]{inputenc}
 \usepackage[czech,english]{babel}
\usepackage{a4wide}
% \usepackage[numbers]{natbib}
%\usepackage{czech}



\title{Funkce repository návrhu systému a důvody pro její existenci. \\ 4IT415 Informační modelování organizací (ZS 2014-2015)}
\author{Tomáš Maršálek}
\date{\today}

\begin{document}
\maketitle
\thispagestyle{empty}
\clearpage
\pagenumbering{gobble}

Repository pattern je návrhový vzor s účelem logického oddělení perzistenční
vrstvy systému od aplikační. Do systému přináší nízkou provázanost (loose
coupling), což je obecně považováno jako dobrá vlastnost při návrhu. Nízká
provázanost umožňuje nahrazení části s minimálním zásahem do jiných částí
systému. Proč bychom ale v první řadě uvažovali o nahrazování perzistenční
vrstvy?

Prvním důvodem může být nahrazení implementace úložiště. Repository nám dává
možnost snadné migrace mezi různými databázemi. Nebo pokud chceme nahradit
pouze část úložiště, např. když chceme ukládat obrázky do souborů a ne přímo do
databáze, aplikační kód se o téhle změně vůbec nemusí dozvědět, protože k
obrázkům přistupuje pouze definovaných metod rozhraní repozitáře.

Změna implementace úložiště je ale většinou uváděna pouze jako ilustrativní
využití repository, protože ke změnám implementace v produkčním systému dochází
zřídkakdy. Většinou existuje datová základna a aplikace se podle ní
přizpůsobují než naopak.

Dalším, a to podstatnějším, důvodem je možnost používat unit testy pro
aplikační kód, který je provázaný perzistenčním kódem, a ne jen integrační
testy, které testují aplikaci jako celek. Repository nám totiž umožňuje
vytvořit takzvanou Mock implementaci perzistenčního rozhraní, díky které je
možné jednoduše vytvořit sadu deterministických unit testů. Unit testy se
skutečnou databází jsou velmi krkolomné, protože před každým testem je třeba
testovací databázi vyprázdnit a deterministicky naplnit daty potřebnými pro
konkrétní unit test.

Velkou výhodou oddělení perzistence od aplikace je cachování v paměti, o kterém
aplikace nemusí vůbec vědět, protože se vše děje na pozadí v perzistenční
vrstvě, která může najednou přistupovat přímo k databázi jako datovému úložišti
a mezitím k dočasnému úložišti v paměti nebo k jiné databázi, která slouží jako
cache.

Repository pattern není často oblíbený, protože není flexibilní. Rozhraní
repository pak může obsahovat metodu pro každou permutaci získání objektu
(např. getUserById, getUserByName, getUsersWithDetails,
getUsersWithDetailsAndFavouriteColorByAddress). Tento problém spadá do většího
problému zvaného object-relational impedance mismatch, který ve zkratce
znamená, že je velmi obtížné přesně namapovat objekty z OO světa do relačního
modelu. Doposud nikdo nepřišel s řešením, které by bylo vhodné pro všechny
případy a stále by mělo vlastnosti nízkého provázání perzistenční a aplikační
vrstvy. Problém dostal přezdívku \uv{Vietnam of Computer Science} v analogii s
Vietnamskou válkou, do které se dlouhé roky investovaly neuvěřitelné peníze a
výsledek nikde. 

Repository je jedním z řešení tohoto problému s vlastností nizkého provázání,
avšak za cenu velmi složitého rozhraní, jak bylo zmíněno výše. Jediným uváděným
řešením jak se zbavit O/R mapování je odstranit z názvu \uv{O} nebo \uv{R},
tedy nepoužívat objekty v aplikaci, anebo nepoužívat relační, ale například
objektovou nebo grafovou NoSQL databázi. Dalšími navrhovanými řešeními jsou
experimentální Event Sourcing (např. Akka persistence ve Scale nebo acid-state
v Haskellu).

Řešení, které se v poslední době s rozšířením architektury MVC pro webové
aplikace stalo populární, je takzvané ORM - Object Relational Mapper, neboli
nástroj, který provede konverzi dat z tabulek relační databáze do grafové
reprezentace objektů v objektově orientovaném programování. Vývojář se nesetká
s neduhy jako u repository, kde musel vytvořit každou permutaci databázového
dotazu, ale tvoří databázové dotazy přímo v kódu podle potřeby. Nevytváří je
ale v jazyku závislém na implementaci databáze, ale v konstrukcích nezávislých
na implementaci. Alespoň tohle byl původní záměr a bohužel skutečnost je jiná.
ORM přináší pouze výhodu v tom, že se v aplikační vrstvě nemusí používat přímo
databázový dotazovací jazyk, ale jeho varianta, která odpovídá konstrukcím
programovacího jazyka aplikační vrstvy. Nedocílíme nízkého provázání, nelze
vytvářet mocky objekty (implementace ORM nástrojů je hutná a vytvořit mock,
který by splňoval jejich rozhraní je velmi obtížné) a cachování dat je obtížné,
protože v některých výjimečných případech musíme obejít ORM vrstvu a
přistupovat přímo do perzistenční.

Každé řešení má své výhody a nevýhody a proto nachází uplatnění na jiných
pozicích.  ORM získalo v poslední době ve webových frameworcích místo mezi
aplikací a perzistencí i přes veliký tábor jeho odpůrců. Repository pattern se
často využívá při návrhu API, který v poslední době silně konverguje k jednomu
řešení - REST (Representational state transfer pro protokol HTTP).

Praktické rady při použití repository patternu. 

Přestože je repository pattern velmi neflexibilní s přibývajicími požadavky na
variaci dotazů na data, můžeme alespoň částečně přidat flexibilitu dotazu, aniž
bychom narušili únik abstrakce skrze vrstvy nebo volné provázání. Rozhraní
repozitáře by mělo být schopné alespoň stránkovat a řadit výsledky.

Jiným návrhem pro návrh repository je takzvané CQRS (Command Query
Responsibility Segregation), které jednoduše odděluje zápisy do perzistence a
čtení do oddělených komponent. To umožňuje použít dva ruzné oddělené objektové
modely nebo náhledy na stejná data, které jsou vhodnější pro čtení nebo zápis
než jednotný model. Výhodou nízkého provázání čtení a zápisu je flexibilita při
nasazení aplikace na distribuovaný systém - např. čtecí repositář může
obsluhovat 10 výpočetních jednotek a pro zápis můžou být pouze 2.

Cílem eseje bylo přiblížit účel repository patternu, jeho výhody a nevýhody a
porovnání s ostatními řešeními problému objektově relačního mapování. Dále jsem
zmínil postavení repository patternu v dnešní době ve skutečných aplikacích a
rady, které vývojová komunita po letech praxe nalezla. Esej se dotýká širokého
rozsahu témat spojených s timto návrhovým vzorem a pro plné pochopení
problematiky doporučuji detailnější literaturu, která je většinou ve formě
blogových příspěvků a zejména diskusí pod nimi.


% \bibliographystyle{csplainnat}
% \bibliography{ref}

\end{document}
